\documentclass[11pt,a4paper]{article}
\author{Tim Gesthuizen}
\usepackage{tikz}
\title{Game 24}
\usetikzlibrary{trees}

\begin{document}

\maketitle

\section{Problem description}

A program shall be written that solves the \textit{Game of 24}.
In this game you are given 4 numbers and need to come up with a
selection of the 4 arithmetic operators $+$, $-$, $\cdot$ and $\div$ to
reduce these numbers to the number 24.
The numbers can be rearranged in any order and the order of the
operations can also be manipulated.
See figure \ref{fig:trees} for examples of syntax trees.

\input{tikz/syntax-trees.inc}

\section{Problem analysis}

The problem is to construct all possible syntax trees and evaluate
them systematically.
Selecting the operators is not difficult:
We have a set $Operators = \{+, -, \cdot, \div\}$ and generate all
permutations of a multi-set $M$ such that $\vert M \vert = 3$.
\newcommand{\complexop}{{\vert Operators \vert}^3}
\newcommand{\opconnect}{4 \cdot 3 \cdot 3 \cdot 2 \cdot 2 \cdot 1}
This results in $\complexop$ combinations of
operators.
The first operators left hand side can now be choosen from one of the
four input operators.
The right hand side can then be choosen from one of the remaining
three numbers.
After the first operator has been ``wired'' it becomes available as an
input for the next operator.
Therefore the total complexity of connecting the operators is $\opconnect$.
This results in
$\complexop \cdot \opconnect = 11664$ possibilities to solve the
problem.

\section{Design of the program}

The user enters the four numbers the program shall find solutions
for. The program then calculates all permutations and builds up the
syntax trees for them. If a syntax tree evaluates to 24, it is
printed.
It should be noted that all of this can be done in $\mathcal{O}(1)$
space.
The above requirements result in the following software parts:
\begin{enumerate}
\item {\tt main} and user interactions
\item Syntax tree creation and permutation
\item Syntax tree evaluation
\item Syntax tree printing
\end{enumerate}

\end{document}